\documentclass[12pt]{article}
\usepackage{amsmath} % needed for the align* environment

\begin{document}
\vspace*{-15mm}
\begin{center}
CSCE 222 Discrete Structures for Computing\\[.5ex]
Hyunyoung Lee\\[1ex]
\textbf{Example \LaTeX\ source for creating a truth table and aligning derivation steps}\\[1ex]
\end{center}

\noindent
For example, to draw the truth table for $p \land q$ and $p \oplus q$ as below
\begin{displaymath}
\begin{array}{|c c||c|c|} 
% |c c|c|c| means that there are four columns in the table where 
% a vertical bar '|' will be printed on the left and right borders,
% and between the third and the fourth columns.  I put double  
% vertical bars '||' between the second and the third columns to 
% separate the possible value combinations for the variables  
% (the first two columns) and the resulting values of the operations  
% on those values (the last two columns). Note that between the 
% first two columns there is no bar.
% 
% The letter 'c' means the value will be centered within the column. 
% If you want the value to be left-aligned, then you give letter 'l' 
%  instead, and to have it right-aligned, give letter 'r'.
%
p & q & p \land q & p \oplus q\\ % Use & to separate the columns
\hline  % Put a horizontal line between the table header and the rest.
T & T & T & F\\
T & F & F & T\\
F & T & F & T\\
F & F & F & F\\
\end{array}
\end{displaymath}

\noindent
the \LaTeX\ source looks like this:
\small{
\begin{verbatim}
\begin{displaymath}
\begin{array}{|c c||c|c|} 
% |c c|c|c| means that there are four columns in the table where 
% a vertical bar '|' will be printed on the left and right borders,
% and between the third and the fourth columns.  I put double  
% vertical bars '||' between the second and the third columns to 
% separate the possible value combinations for the variables  
% (the first two columns) and the resulting values of the operations  
% on those values (the last two columns). Note that between the 
% first two columns there is no bar.
% 
% The letter 'c' means the value will be centered within the column. 
% If you want the value to be left-aligned, then you give letter 'l'  
% instead, and to have it right-aligned, give letter 'r'.
%
p & q & p \land q & p \oplus q\\ % Use & to separate the columns
\hline  % Put a horizontal line between the table header and the rest.
T & T & T & F\\
T & F & F & T\\
F & T & F & T\\
F & F & F & F\\
\end{array}
\end{displaymath}
\end{verbatim}
}

\goodbreak

To align derivation steps, we can use the align or align* environment.  
The align environment puts a label at the end of each line whereas align* 
does not.  Also note that the align environment is already a math 
environment, i.e., within the align environment, you do not enclose math 
symbols within the dollar signs. For example, to show the derivation of the 
logical equivalence $\neg(p\rightarrow q)\equiv p\land \neg q$ in the last 
page of lecture slides propositional.pdf, you can do as below.
\begin{align*}
% Each line of derivation must end with two backslashes \\ (newline 
% symbol in LaTeX).
% In each line, there must be one ampersand & symbol preceding the 
% symbol that you want to align.  In this example, the symbol to be 
% aligned is the equivalence symbol \equiv, thus we put the & symbol 
% right before the \equiv symbol.
% The \quad command puts some space there.
% \mbox{ } is to display a text within the math environment.  It is used 
% when you need to put explanations within the math mode.
%
\neg(p\rightarrow q) 
&\equiv \neg(\neg p\lor q) \quad \mbox{ previous result: } 
                           p\rightarrow q\equiv \neg p\lor q\\
&\equiv \neg(\neg p) \land \neg q \quad \mbox{ de Morgan's Law}\\
&\equiv p\land \neg q \quad \mbox{ double negation law}
\end{align*}
\noindent
for which, the \LaTeX\ source looks like this:
\small{
\begin{verbatim}
\begin{align*}
% Each line of derivation must end with two backslashes \\ (newline 
% symbol in LaTeX).
% In each line, there must be one ampersand & symbol preceding the 
% symbol that you want to align. In this example, the symbol to be 
% aligned is the equivalence symbol \equiv, thus we put the & symbol 
% right before the \equiv symbol.
% The \quad command puts some space there.
% \mbox{ } is to display a text within the math environment. It is used 
% when you need to put explanations within the math mode.
%
\neg(p\rightarrow q) 
&\equiv \neg(\neg p\lor q) \quad \mbox{ previous result: } 
                           p\rightarrow q\equiv \neg p\lor q\\
&\equiv \neg(\neg p) \land \neg q \quad \mbox{ de Morgan's Law}\\
&\equiv p\land \neg q \quad \mbox{ double negation law}
\end{align*}
\end{verbatim}
\end{document}
